%==================================================================
% Ini adalah bab 2
% Silahkan edit sesuai kebutuhan, baik menambah atau mengurangi \section, \subsection
%==================================================================

\chapter[PENDEKATAN PEMECAHAN MASALAH]{\\ PENDEKATAN PEMECAHAN MASALAH}

\section{Bedengan}
Bedengan adalah gundukan tanah yang dibuat untuk menanam tanaman dengan lebar dan tinggi tertentu \cite{kurniawan2024pengaruh}. Pada umumnya, bedengan dibuat dengan ukuran lebar 70-120 cm atau lebih, dan tinggi 20-30 cm dengan panjang yang bervariasi. Bedengan berfungsi untuk membuat tanah lebih gembur, sebagai saluran irigasi, dan memudahkan petani dalam melakukan penanaman, perawatan, hingga panen \cite{hariatimodifikasi} berikut ini adalah \cref{fig:bedengan}. 
\begin{figure}[H]
	\centering
	\includegraphics[scale=0.2]{row-crop}
	\caption{Bedengan}
	\label{fig:bedengan}
\end{figure}

\section{YOLOv8}
\begin{figure}[H]
	\centering
	\includegraphics[scale=0.4]{hirarkiAi}
	\caption{Hubungan antara tiga konsep utama dalam bidang kecerdasan buatan}
	\label{fig:hirarki}
\end{figure}
Yolov8 (You only Look Once)  versi 8 merupakan salah satu teknologi atau metode pendekatan dilingkup deep learning yang merupakan sub-bidang bagian dari keseluruhan ekosistem Artificial Intelligence seperti pada \cref{fig:hirarki}. \textit{Yolov8} menggunakan Convolutional Neural Networks(CNN) untuk memproses gambar dan mendeteksi objek di dalamnya. YOLOv8 dikembangkan oleh Ultralytics, merupakan model deteksi objek yang mempunyai dukungan out-of-the-box untuk tugas-tugas seperti deteksi objek, klasifikasi, dan segmentasi. YOLOv8 memiliki keunggulan kecepatan dan akurasi untuk memproses gambar secara real-time \cite{roboflowWhatYOLOv8}. Arsitektur YOLOv8 mempunyai peningkatan dengan fokus pada akurasi dan kecepatan yang lebih baik dibandinkan dengan versi sebelumnya. berikut ini adalah 3 komponen penting dari YOLOv8:
\begin{packed_item}
	\item \textbf{Backbone:}
	Backbone adalah  jaringan saraf konvolusi (CNN) yang bertugas untuk mengekstraksi fitur dari gambar masukan. YOLOv8 menggunakan custom CSPDarknet53 sebagai backbone,  CSPDarknet53 merupakan varian dari  arsitektur Darknet53 yang telah dioptimalkan yang dapat meningkatkan aliran informasi antar lapisan dan meningkatkan akurasi.
	\item \textbf{Neck:} 
	\textit{Neck} dikenal sebagai \textit{feature extractor} berfungsi untuk peta fitur dari berbagai tahap di \textit{backbone} untuk menangkap informasi pada berbagai skala. \textit{YOLOv8} menggunakan \textit{C2f} sebagai pengganti \textit{Feature Pyramid Network (FPN)} yang dapat lebih baik dalam mengenali objek dengan berbagai ukuran.  
	\item \textbf{Head:}
	\textit{Head} mempunyai tugas utama untuk membuat prediksi. \textit{YOLOv8} menggunakan beberapa modul deteksi yang memprediksi \textit{Bounding Box, Objectness Scores, dan Class Probabilities}
\end{packed_item}	

Berikut ini terdapat  arsitektur dari YOLOv8 yang tersaji pada \cref{fig:yolov8}.
\begin{figure}[H]
	\centering
	\includegraphics[scale=0.4]{yolov8-architecture}
	\caption{Arsitektur YOLOv8}
	\label{fig:yolov8}
\end{figure}

YOLOv8 menggunakan pendekatan \textit{ "Anchor Free Detection"} yang menghilangkan kebutuhan akan anchor boxes yang telah diatur sebelumnya untuk membuat prediksi bounding box. Sebagai gantinya, model belajar untuk mendeteksi objek langsung dari fitur gambar, yang menyederhanakan arsitektur dan mengurangi kompleksitas komputasi. Pendekatan tersebut mempunyai keuntungan lebih efisien karena Mengurangi jumlah prediksi yang harus diproses, sehingga mempercepat inferensi dan mengurangi penggunaan memori. Selain itu mempunyai akurasi yang lebih tinggi karena dapat Menghindari kesalahan prediksi yang disebabkan oleh anchor boxes yang tidak cocok dengan ukuran dan bentuk objek sebenarnya. Berikut ini \cref{fig:anchor} adalah visualisasi dari sebuah anchor box dalam arsitektur YOLO.

\begin{figure}[H]
	\centering
	\includegraphics[scale=0.6]{AnchorYolo}
	\caption{Visualisasi anchor box YOLO}
	\label{fig:anchor}
\end{figure}

\section{Fuzzy Logic}
Fuzzy Logic diperkenalkan pada tahun 1965 oleh Professor Lotfi A. Zadeh melalui karyanya yang berjudul "Fuzzy Sets". Fuzzy Logic adalah sebuah metode dalam ilmu komputer dan matematika yang menangani konsep kebenaran yang tidak bersifat biner atau absolut.  \cite{kambalimath2020basic}.Metode ini memungkinkan pemetaan ruang input ke dalam ruang output dengan nilai yang kontinu dan fleksibel. Dalam logika fuzzy, nilai-nilai dinyatakan dalam derajat keanggotaan dan derajat kebenaran, yang memungkinkan suatu hal dapat dinyatakan sebagian benar dan sebagian salah secara bersamaan. Hal ini menjadikan Fuzzy Logic sangat efektif untuk menangani ketidakpastian dan ambiguitas dalam berbagai aplikasi, seperti sistem kontrol, pengambilan keputusan, dan kecerdasan buatan.\cite{sri2003artificial}. Berikut ini adalah beberapa konsep kunci pada fuzzy logic:
\subsection{\textit{Himpunan Crisp Dan Himpunan Fuzzy}}
Himpunan Crisp(himpunan tegas) adalah himpunan dimana setiap elemen hanya memiliki 2 kemungkinan yaitu termasuk keanggotaan himpunan atau tidak mejadi keanggotaan himpunan. Funsgi keanggotaan mendefinisikan apakah suatu elemen x alalah anggota himpunan. Jika elemen tersebut adalah anggota himpunan maka x=1, jika bukan bagian dari keanggotaan maka x=0. Sedangkan Himmpunan Fuzzy nilai x berada di antara nilai 0 dan 1, nilai 0 menunjukkan elemen tersebut bukan anggota himpunan dan 1 menunjukkan elemen tersebut merupakan anggota himpunan\cite{rindengan2019sistem}. 
\subsection{\textit{Fungsi Keanggotaan}}
Fungsi keanggotaan(membership function) adalah komponen penting yang memungkinkan representasi derajat keanggotaan elemen dalam himpunan fuzzy. Fungsi keanggotaan dapat dapat menangani  ketidakpastian dan ambiguitas dalam berbagai aplikasi, termasuk sistem kontrol, pengambilan keputusan, dan pengenalan pola. Fungsi keanggotaan memiliki 3 jenis yaitu, fungsi segitiga, trapesium, dan Gaussian.
\subsection{\textit{Sistem Interferensi Fuzzy}}
Sistem Interferensi Fuzzy (Fuzzy Inference System)  adalah framework untuk melakukan  penalaran atau inferensi berdasarkan logika fuzzy. Sistem ini berfungsi untuk mengambil keputusan menggunakan aturan-aturan fuzzy dan input yang telah difuzzifikasi.  Fuzzy Inference System memiliki komponen utama yaitu, Fuzzifikasi, Basis Aturan Fuzzy, dan 	Defuzzifikasi, Inference Engine, Agregasi Output Fuzzy, dan Defuzzifikasi.
\subsection{\textit{Metodologi Desain Sistem Fuzzy}}
Merupakan  pendekatan sistematis untuk merancang dan mengimplementasikan sistem berbasis logika fuzzy. 
\subsection{\textit{Metode Mamdani}}
Pendekatan metode mamdani populer digunakan dalam sistem inferensi fuzzy untuk mengimplementasikan logika fuzzy karenan kemampuanya menangani ketidakpastian dan ambiguitas dengan cara yang intuitif dan mudah dipahami. 

\section{Google Colab}
\begin{figure}[H]
	\centering
	\includegraphics[scale=0.6]{GoogleColab}
	\caption{Google Colaboratory}
	\label{fig:Google-Colaboratory}
\end{figure}

Pada \cref{fig:Google-Colaboratory} merupakan logo \textit{Google Colab} atau \textit{Google Colaboratory}, merupakan platform yang sangat populer berbasis cloud untuk membuat dan menjalankan kode Python secara interaktif di dalam notebook. Memiliki fitur kemudahan mengakses pemrosesan grafis (GPU) dan tensor processing unit (TPU) yang sangat berguna untuk melatih model machine learning berskala besar dan kompleks. Google Colab juga mendukung berbagai library python, antara lain  NumPy, Pandas, TensorFlow, PyTorch, dan banyak lainnya. Sehingga memudahkan untuk melakukan analisis data, machine learning, dan deep learning langsung dalam notebook. 

\section{Roboflow}
\begin{figure}[H]
	\centering
	\includegraphics[scale=0.1]{logo-roboflow}
	\caption{Roboflow}
	\label{fig:roboflow}
	
\end{figure}
Roboflow adalah platform yang memfasilitasi proses pengembangan, pelatihan model computer vision. Roboflow dirancang membuat tugas-tugas yang berhubungan dengan pengenalan gambar menjadi lebih mudah dan efisien. Berikut ini adalah fitur-fitur utama Roboflow:
\begin{packed_item}
	\item \textit{Annotasi Data}: Roboflow menyediakan alat untuk anotasi data gambar. Roboflow dapat menandai objek dalam gambar dengan berbagai jenis anotasi seperti bounding box, polygon, dan segmentation mask. Hal ini bertujuan untuk membangun  \textit{dataset} yang berkualitas untuk model \textit{computer vision}.
	
	\item Pengelolaan \textit{Dataset}
	Roboflow memungkinkan untuk mengunggah, mengelola, dan mengorganisir dataset gambar dan mengelompokkan, menggabungkan, dan memodifikasi dataset sesuai kebutuhan.
	
	\item \textit{Augmentasi Data}
	Roboflow memberikan berbagai teknik augmentasi data yang dapat digunakan untuk meningkatkan ukuran dan keragaman dataset tanpa memerlukan gambar tambahan. 

	\item \textit{Preprocessing} dan \textit{Ekspor}
	FItur ini berfungai untuk melakukan preprocessing pada gambar untuk memastikan bahwa data dalam format yang optimal untuk pelatihan model. Roboflow mendukung ekspor dataset dalam berbagai format yang kompatibel dengan berbagai framework machine learning seperti TensorFlow, PyTorch, Darknet (YOLO), dan lainnya.
\end{packed_item}

\section{NVIDIA Jetson Nano}
NVIDIA Jetson Nano adalah produk dari NVIDIA yang dirancang khusus untuk keperluan pengembangan dan implementasi solusi Internet of Things (IoT) dengan kemampuan komputasi terdiri dari CPU dengan kecepatan 1,43 GHz dan GPU dengan 128 inti dari generasi Maxwell. NVIDIA Jetson Nano dapat digunakan untuk machine learning, computer vision. Selain itu dapat digunakan dengan berbagai framework pengembangan AI, seperti Pandas, Numpy, Tensorflow, Keras, PyTorch, dan MXNet. Berikut ini adalah gambar \cref{fig:jetson} NVIDIA Jetson Nano.

\begin{figure}[H]
	\centering
	\includegraphics[scale=0.2]{Jetson}
	\caption{NVIDIA Jetson Nano}
	\label{fig:jetson}
\end{figure}

NVIDIA Jetson Nano memiliki pin General Purpose Input/Output (GPIO) untuk menghubungkan berbagai sensor dan perangkat eksternal. NVIDIA Jetson Nano juga mendukung koneksi kamera eksternal melalui antarmuka CSI atau kamera USB. NVIDIA Jetson Nano memiliki spesifikasi teknis yang ditunjukkan dalam \cref{tab:jetson-nano}	

\begin{table}[H]
	\caption{Spesifikasi NVIDIA Jetson Nano}
	\label{tab:jetson-nano}
	\centering
	\begin{tabular}{|l|l|}
		\hline
		\multicolumn{1}{|c|}{Supply} & 5V 2A (micro USB) / 5V 4A (jack DC)                                \\ \hline
		GPU                          & 128-core Maxwell                                                   \\ \hline
		CPU                          & Quad-core ARM A57 @ 1.43 GHz                                       \\ \hline
		Memory                       & 4 GB 64-bit LPDDR4 25.6 GB/s                                       \\ \hline
		Storage                      & microSD                                                            \\ \hline
		Video encode                 & 4K @ 30 | 4x 1080p @ 30 | 9x 720p @ 30 (H.264/H.265)               \\ \hline
		Video decode                 & 4K @ 60 | 2x 4K @ 30 | 8x 1080p @ 30 | 18x 720p @ 30 (H.264/H.265) \\ \hline
		Camera                       & 2x MIPI CSI-2 DPHY lanes                                           \\ \hline
		Connectivity                 & Gigabit Ethernet, M.2 Key E                                        \\ \hline
		Display                      & HDMI and display port                                              \\ \hline
		USB                          & 4x USB 3.0, USB 2.0 Micro-B                                        \\ \hline
		I/O                          & GPIO, I2C, I2S, SPI, UART                                          \\ \hline
		Mechanical                   & 69 mm x 45 mm, 260-pin edge connector                              \\ \hline
	\end{tabular}
\end{table}

\section{Doit ESP32 DevKit V1}
Doit ESP32 DevKit V1 adalah salah satu dari berbagai pengembangan perangkat keras (hardware development) yang didasarkan pada mikrokontroler ESP32 yang sangat populer. ESP32 merupakan produk dari Espressif Systems, yang memiliki kemampuan Wi-Fi dan Bluetooth yang kuat serta berbagai fitur lainnya yang membuatnya sangat cocok untuk berbagai proyek IoT (Internet of Things). Berikut ini adalah gambar \cref{fig:ESP} yang menampilkan ESP32 DevKit V1.

\begin{figure}[H]
	\centering
	\includegraphics[scale=0.1]{doitesp32}
	\caption{Doit ESP32 DevKit V1}
	\label{fig:ESP}
\end{figure}

DevKit V1 adalah salah satu varian dari seri pengembangan ESP32 yang menyediakan berbagai fitur dan konektivitas yang berguna bagi para pengembang. Beberapa fitur utamanya termasuk:
\begin{packed_item}
	\item Mikrokontroler ESP32: DevKit V1 dilengkapi dengan mikrokontroler ESP32 yang memiliki CPU dual-core Xtensa 32-bit, serta RAM dan Flash yang cukup untuk menjalankan aplikasi yang kompleks.
	\item Modul Wi-Fi dan Bluetooth: ESP32 memiliki kemampuan terintegrasi untuk Wi-Fi dan Bluetooth, memungkinkan perangkat untuk terhubung ke jaringan nirkabel dan berkomunikasi dengan perangkat lainnya.
	\item Port I/O: DevKit V1 menyediakan sejumlah pin I/O (Input/Output) yang dapat digunakan untuk berbagai tujuan, termasuk sensor, aktuator, dan antarmuka dengan perangkat lain.
	\item USB-to-Serial Converter: Dilengkapi dengan konverter USB-to-Serial yang memungkinkan pengguna untuk memprogram dan berkomunikasi dengan mikrokontroler melalui koneksi USB.
	\item JTAG Inference: DevKit V1 sering kali memiliki  JTAG Inference yang memungkinkan debugging dan pengembangan yang lebih lanjut menggunakan perangkat lunak seperti OpenOCD atau platform pemrograman terpadu (IDE) seperti PlatformIO.
	\item Bentuk dan Faktor Bentuk yang Sesuai: DevKit V1 dirancang dengan bentuk dan faktor bentuk yang umum digunakan dalam pengembangan perangkat keras, sehingga dapat dengan mudah diintegrasikan ke dalam proyek-proyek yang ada.
\end{packed_item}

Berikut ini adalah \cref{tab:spesifikasi} yang memuat spesifikasi ESP32 DevKit V1

\begin{table}[H]
	\caption{Spesifikasi ESP32 DevKit V1}
	\label{tab:spesifikasi}
	\centering
	\begin{tabular}{|l|l|}
		\hline
		\multicolumn{1}{|c|}{Number of Core} & 2 (Dual Core)                                                                                                                                                                                                                                                                \\ \hline
		Wi-Fi                                & 2.4 GHz up to 150 Mbit/s                                                                                                                                                                                                                                                     \\ \hline
		Bluetooth                            & BLE (Bluetooth Low Energy) and legacy Bluetooth                                                                                                                                                                                                                              \\ \hline
		Architecture                         & 32 bits                                                                                                                                                                                                                                                                      \\ \hline
		Clock frequency                      & up to 240 MHz                                                                                                                                                                                                                                                                \\ \hline
		RAM                                  & 512 KB                                                                                                                                                                                                                                                                       \\ \hline
		Pins                                 & 30                                                                                                                                                                                                                                                                           \\ \hline
		Peripherals                          & \begin{tabular}[c]{@{}l@{}}Capacitive touch, Analog to Digital Converter, \\ Digital to Analog converter, I2C, UART, \\ Controller Area Network,  Serial Peripheral Interface, \\ Integrated Inter-IC Sound, \\ Reduced Media-Independent Interface, \\ and PWM\end{tabular} \\ \hline
	\end{tabular}
\end{table}

\section{\textit{Webcam}}
Webcam Berfungsi untuk merekam vidio atau gambar secara real-time yang dapat diakses menggunakan world wide web (www), program instant messaging, atau aplikasi komunikasi dengan tampilan video pada PC untuk berbagai tujuan, diantaranya sistem keamanan, pengenalan wajah, vidio konfersi dll. Webcam pada umumnya terhubung ke komputer melalui USB atau koneksi nirkabel. Webcam telah menjadi yang sangat umum dan terintegrasi dalam laptop, tablet, dan bahkan beberapa ponsel pintar. Berikut ini adalah \cref{fig:webcam} Webcam Fantech Luminous C30. 

\begin{figure}[H]
	\centering
	\includegraphics[scale=0.2]{webcam}
	\caption{Gambar Webcam Fantech Luminous C30}
	\label{fig:webcam}
\end{figure}

Berikut ini \cref{tab:spesifikasi-webcam} yang berisi spesifikasi dari  Webcam Fantech Luminous C30

\begin{table}[H]
	\caption{Spesifikasi Webcam Fantech Luminous C30}
	\label{tab:spesifikasi-webcam}
	\centering
	\begin{tabular}{|l|l|}
		\hline
		\multicolumn{1}{|c|}{Resolution} & 2560 x 1440               \\ \hline
		Field of view                    & 106° (ultra wide)         \\ \hline
		Pixels                           & 4 Megapixel               \\ \hline
		Frame Rate                       & 30fps (1080p), 25fps (2K) \\ \hline
		Microphone                       & Built-in microphone       \\ \hline
		Interface Type                   & USB 2.0                   \\ \hline
		Focusing Range                   & 90 cm                     \\ \hline
		Flashing Control                 & 50Hz                      \\ \hline
		Dimension                        & 73x26x33mm                \\ \hline
		Weight                           & 87gr                      \\ \hline
	\end{tabular}
\end{table}

\section{Driver Motor IBT-2 BTS7960}
Driver Motor IBT-2 BTS7960 adalah modul driver motor yang sering digunakan dalam aplikasi kendali motor DC (Direct Current) dengan arus besar. IBT-2 BTS7960 adalah jenis driver motor yang memiliki kemampuan untuk mengendalikan arah putaran dan kecepatan motor DC dengan menggunakan sinyal PWM (Pulse Width Modulation).
Driver motor ini menggunakan chip BTS7960 sebagai bagian intinya. Chip ini mampu menangani arus hingga beberapa puluhan ampere dengan efisiensi yang tinggi. Modul IBT-2 BTS7960 memiliki fitur perlindungan termal dan arus berlebih, sehingga dapat memberikan keamanan tambahan pada sistem kendali motor. Beriku ini adalah \cref{fig:driver-motor-ibt} Driver Motor IBT-2 BTS796.

\begin{figure}[H]
	\centering
	\includegraphics[scale=0.2]{driver-motor-IBT}
	\caption{Driver Motor IBT-2 BTS7960}
	\label{fig:driver-motor-ibt}
\end{figure}

\section{Motor DC MY1016Z-250W 24V}
Motor DC MY1016Z-250W 24V adalah jenis motor DC (Direct Current). Motor ini sering digunakan dalam kendaraan listrik seperti sepeda listrik, skuter listrik, dan kendaraan kecil lainnya. Terdiri dari bagian-bagian seperti kumparan, komutator, dan rotor yang terpasang pada poros. Konstruksi ini memungkinkan motor untuk mengubah energi listrik menjadi gerakan mekanis. Kemampuan motor untuk memberikan torsi atau gaya putar. Torsi yang dihasilkan oleh motor bergantung pada berbagai faktor termasuk arus yang digunakan, tegangan, dan desain mekanis. Berikut ini adalah \cref{fig:motordc} dan spesifikasi yang termuat pada \cref{tab:spesifikasi-motor}

\begin{figure}[H]
	\centering
	\includegraphics[scale=0.2]{motor-dc}
	\caption{Motor DC MY1016Z-250W 24V}
	\label{fig:motordc}
\end{figure}

\begin{table}[H]
	\caption{Spesifikasi Motor DC MY1016Z-250W 24V}
	\label{tab:spesifikasi-motor}
	\centering
	\begin{tabular}{|ll|}
		\hline
		\multicolumn{2}{|c|}{Model  MY1016Z2}            \\ \hline
		\multicolumn{1}{|l|}{Rated Power}      & 250W    \\ \hline
		\multicolumn{1}{|l|}{Rated voltage}    & 24VDC   \\ \hline
		\multicolumn{1}{|l|}{rated speed}      & 3000RPM \\ \hline
		\multicolumn{1}{|l|}{Cut-in-speed}     & 3850RPM \\ \hline
		\multicolumn{1}{|l|}{Rated current}    & $\leq$ 13.3A \\ \hline
		\multicolumn{1}{|l|}{No load current}  & $\leq$ 2.2A  \\ \hline
		\multicolumn{1}{|l|}{Rated torque}     & 0.80Nm  \\ \hline
		\multicolumn{1}{|l|}{Motor efficiency} & $\geq$ 78\%  \\ \hline
		\multicolumn{1}{|l|}{Raduction ratio}  & 1:9.78  \\ \hline
	\end{tabular}
\end{table}
