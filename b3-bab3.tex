%==================================================================
% Ini adalah bab 3
% Silahkan edit sesuai kebutuhan, baik menambah atau mengurangi \section, \subsection
%==================================================================

\chapter[KONSEP RANCANGAN / PRODUK / JASA / EVALUASI / PENGUJIAN]{\\ KONSEP RANCANGAN PRODUK DAN PENGUJIAN}

\section{Waktu dan Tempat Penelitian}
Penelitian tugas akhir berjudul "Rancang Bangun Sistem Navigasi Otonom Pada Robot Semprot Pestisida menggunakan YOLOv8 dan Fuzzy Logic Controller" dimulai sejak bulan Juli hingga November 2023. Penelitian ini dilaksanakan di Universitas Negeri Yogyakarta Kampus Wates, yang berada Kabupaten Kulon Progo, Daerah Istimewa Yogyakarta dan Dusun Terbah, Desa Terbah, Kecamatan Patuk, Kabupaten Gunungkidul, Daerah Istimewa Yogyakarta.

\section{Tahapan Penelitian}
Pada penelitian ini terdapat terdapat beberapa tahapan yang digambarkan pada  \cref{fig:flowchart} flowchart dibawah ini. 

\begin{figure}[H]
	\centering
	\includegraphics[scale=0.6]{tahap_penelitian}
	\caption{Flowchart Tahap Penelitian}
	\label{fig:flowchart}
\end{figure}

\begin{packed_enum}
	\item Studi Literatur: Pada tahap ini dilakukan pencarian dan analisis terhadap literatur yang relevan dengan topik atau latar belakang penelitian. Tahap ini bertujuan untuk menemukan dasar teori, memahami metode penelitian sebelumnya untuk dikembangkan sesuai topik penelitian, Mengumpulkan data dan bukti pendukung untuk mendukung argumen dan analisis dalam penelitian.
	\item Identifikasi Kebutuhan: Identifikasi kebutuhan diperlukan untuk menganalisa dan mempersiapkan kebutuhan yang dibutuhkan pada penelitian. Pada tahap ini identifikasi kebutuhan dibagai menjadi 3 yaitu: kubutuhan hardware, kebutuhan software, dan alat.
	\item Perancangan Hardware dan Software: Tahap Rancangan hardware dibagi menjadi dua yaitu rancangan mekanik dan elektronik. Rancangan mekanik maupun elektronik dilakukan dengan menggunakan perangkat lunak CAD (Computer-Aided Design) dan KiCad. Rancangan elektronik meliputi pemasangan atau pemenuhan dependencies library yang dibutuhkan oleh sistem.
	\item Integrasi Sistem: merupakan proses menggabungkan sistem hardware dab software yang telah dirancang dan dikembangkan secara terpisah menjadi satu kesatuan yang dapat berfungsi penuh.
	\item Pengujian Sitem: proses mengevaluasi sistem yang telah diintegrasikan untuk memastikan komponen berfungsi sesuai dengan spesifikasi dan persyaratan yang telah ditentukan.
\end{packed_enum}

\section{Identifikasi Kebutuhan}
Tahap penelitian setelah melakukan studi literatur adalah identifikasi kebutuhan untuk memenuhi kebutuhan sumber daya yang diperlukan sesuai dengan yang dibutuhkan. Identifikasi kebutuhan pada penelitian ini terbagi menjadi 3: yaitu kebutuhan hardware, software dan kebutuhan alat.

\subsection{Kebutuhan Hardware Mekanik}
\begin{packed_item}
	\item Aluminium Profile: Menggunakan aluminium profile ukuran 3030 dalam pembuatan rangka utama robot, dan ukuran 2020 untuk kerangka penunjang seperti sistem pada penyemprotan. Material ini dipilih karena ringan, tahan korosi dan mudah dibentuk sesuai kebutuhan.  
	\item Sistem Roda: Menggunakan 4 Roda utama sebagai penggerak, roda menggunakan ban dalam dan luar ATV Ring 6 - 4.10 offroad, Velg ATV Ring 6. Komponen-komponen penggerak tersebut dipilih karena dapat memberikan mobilitas yang baik di medan berupa tanah.
	\item Gear dan Rantai: Menggunakan gear Motor Mini GP - 51T dan Rantai Mini GP 25H, material tersebut dipilih untuk mentrasnmisikan daya yang dihasilkan motor DC.
	\item Junction Box PVC:  terbuat dari bahan polivinil klorida (PVC) yang digunakan untuk menyimpan dan melindungi  komponen-komponen elektronik dari debu dan air.
\end{packed_item}

\subsection{Kebutuhan Hardware Elektronik}
\begin{packed_item}
	\item Motor DC  MY1016Z 250W 24V
	\item Driver BTS7960: Driver motor ini dipilih karena dapat mengendalikan motor DC dengan arus maksimum 43A dan tegangan 5V - 27V. Selain itu  Driver BTS7960 mempunyai mode operasi H-bridge yang memungkinkan untuk mengendalikan arah dan kecepatan motor menggunakan sinyal PWM. 
	\item Baterai LiFePO4 24V 20Ah dan LiPO 11.1V 3300Ah
	\item ESP32 DevKit V1: merupakan board pengembangan berbasis mikrokontroler, digunakan karena memiliki banyak pin I/O dan kemampuan ADC, sehingga dapat terhubung dengan berbagai sensor atau modul. 
	\item NVIDIA Jetson Nano: merupakan board pengembangan kecil, dipilih karena kemampuanya dibidang AI. Selain itu NVIDIA Jetson Nano mampu memproses data seperti sistem deteksi YOLO.
	\item  USB Webcam Fantech Luminous C30: Dipilih karena mempunyai fitur resolusi video yang cukup baik, sehingga cukup baik  digunakan pada robot semprot untuk menangkap video secara real-time.
\end{packed_item}

\subsection{Kebutuhan Software}
\begin{packed_item}
	\item Visual Studio Code: Penggunaan  Visual Studio Code karenan mempunyai dukungan multi bahasa pemrograman, terintegrasi dengan terminal, sehingga teks iditor ini dapat digunakan untuk mengembangkan sistem pada robot. 
	\item Arduino IDE: Autodesk Inventor adalah perangkat lunak CAD (Computer-Aided Design) yang digunakan untuk merancang dan memodelkan komponen robot dan sistem mekanis secara detail. Dengan fitur pemodelan 3D yang canggih, Autodesk Inventor memungkinkan pengguna untuk membuat desain yang akurat dan terperinci dari komponen mekanis serta rakitan. Perangkat lunak ini mendukung pembuatan model 3D, simulasi, dan analisis, sehingga mempermudah proses perancangan dan memastikan bahwa desain robot dapat diimplementasikan dengan presisi yang tinggi.
	\item Remmina: Merupakan remote desktop client open source yang berjalan di platform Linux. Software ini dipilih karena mendukung beberapa protokol jaringan untuk menghubungkan dan memanajemen desktop jarak jauh termasuk RDP (Remote Desktop Protocol), SSH (Secure Shell) sehingga software ini  merupakan pilihan yang tepat untuk memanejemen NVIDIA Jetson Nano.
\end{packed_item}

\begin{packed_item}
	\item Laptop: Laptop merupakan alat utama yang digunakan untuk mengembangkan robot semprot. Laptop digunakan untuk mendesain mekanik dan software robot agar dapat instegrasi sesuai dengan sepesifikasi dan fungsi yang telah ditentukan.
	\item Meisn Gerinda Potong: Mesin Gerinda yang digunakan adalah mesin gerinda duduk. Mesin gerinda ini digunakan untuk memotong aluminium profile dengan hasil yang presesi dan sesuai ukuran.
	\item Solder: Digunakan untuk proses wiring pada sistem elektronik.
	\item Multimeter: Alat ukur ini digunakan untuk mengecek sistem elektronik dari aspek keamanan dan untuk mengukur parameter-parameter elektronik sesuai dengan spesifikasi yang telah ditentukan.
	\item Obeng Set Toolkit: Toolkit ini digunakan untuk membantu pembuatan sistem mekanik dan elektronik .
\end{packed_item}

\section{Rancangan Hardware}
Tahap rancangan hardware pada penelitian ini dibagi menjadi dua bagian utama. Rancangan mekanik yang mencakup desain fisik robot, termasuk struktur, sasis, dan mekanisme gerakannya. Rancangan elektronik yang meliputi desain arsitektur sistem, mencakup komponen-komponen elektronik seperti mikrokontroler, sensor, aktuator, dan sistem komunikasi. Kombinasi kedua rancangan ini diharapkan menghasilkan robot sesuai dengan fungsional.

\subsection{Rancangan Mekanik}
Rancangan mekanik merupakan proses perencanaan dan desain yang berfokus pada aspek fisik dan struktural robot. Pada \cref{fig:mekanik-utama} di bawah ini ditunjukkan desain dari robot semprot yang dibuat menggunakan Autodesk Inventor. Struktur atau rangka utama robot menggunakan aluminium profil dengan ukuran 30x30 mm, sementara aluminium profil dengan ukuran 20x20 mm digunakan untuk rangka penunjang seperti tiang sprayer dan tempat jerigen. Robot ini memiliki dimensi panjang 750 mm dan lebar 600 mm, serta dilengkapi dengan empat roda penggerak. Pergerakan robot dihasilkan oleh dua motor DC dengan daya 250W dan tegangan 24V, yang memungkinkan robot untuk bergerak dengan stabil dan efisien. Untuk bekerja di medan tanah, robot menggunakan empat roda utama sebagai penggerak. Roda tersebut menggunakan ban dalam dan luar ATV Ring 6 - 4.10 offroad dengan velg ATV Ring 6.
\begin{figure}[H]
	\centering
	\includegraphics[scale=0.4]{main_mekanik}
	\caption{Gambar Desian Mekanik Robot}
	\label{fig:mekanik-utama}
\end{figure}

Robot didesain mempunyai tiang sprayer dengan tinggi 75 hingga 200 cm dengan total empat nozzle dua nozzle di sisi kanan dan kiri. Cairan pestisida yang ditampung di jerigen disalurkan menuju sprayer menggunakan pompa dan selang pneumatic. Robot dilengkapi dengan junction box untuk menyimpan dan melindungi komponen-komponen elektronik dari debu dan air.

\subsection{Rancangan Elektronik}
Rancangan elektronik robot semprot terdiri dari mikrokontroler, sensor, dan aktuator. Berdasarkan \cref{fig:arsitektur-diagram} terdapat power supply dari Baterai LiFePO4 24V 20Ah sebagi sumber tegangan utama untuk Motor DC, dan juga sumber tegangan untuk komponen-komponen elektronik. ESP32 sebagai mikrokontroler utsms menerima dan mengirimkan data, Jetson Nano yang terhubung denegan ESP32 melalui port USB untuk komunikasi serial. ESP32 juga terhubung dengan Receiver Ibus dari Radio Control. ESP32 pada rancangan elektronik robot semprot digunakan untuk memproses data pengolahan citra sebagi input untuk Fuzzy Logic Controller, output deffuzy berupa nilai PWM yang akan menggerak Motor DC. 

\begin{figure}[H]
	\centering
	\includegraphics[scale=0.4]{arsitektur-diagram}
	\caption{Gambar Arsitektur Rancangan Elektronik}
	\label{fig:arsitektur-diagram}
\end{figure}
 
\section{Rancangan Software}
Perancangan perangkat lunak dalam penelitian ini memiliki peranan penting dalam memaksimalkan kemampuan dan kinerja perangkat keras yang digunakan. Dengan merancang perangkat lunak yang andal, robot semprot dapat beroperasi secara lebih efektif, aman, dan efisien, sehingga siap untuk beroperasi di lapangan.

\subsection{Rancangan Diagram Alir Program}
Pada rancangan diagram alir program (\textit{flowchart}) sangat penting dalam pengembangan software dan sistem. Diagram alir program digunakan dalam perancangan, pengembangan, dan komunikasi. Rancangan diagram alir pada sistem robot semprot dapat diamati pada gambar dibawah ini.
 
\subsection{Rancangan Fuzzy Logic Controller}
Rancangan fuzzy Logic Contrroler mempunyai peranan penting dalam navigasi robot semprot karena kemampuannya untuk menangani ketidakpastian dan variabilitas dalam lingkungan yang dinamis. Rancangan fuzzy Logic Controller menggunakan metode pendekatan mamdani seperti pada gambar dibawah ini dengan bantuan Matlab. Fuzzy Logic Controller terdiri dari dua input dan output, input terdiri dari error, delta error serta output berupa nilai duty cycle motor kanan dan kiri. Nilia error diperoleh dari hasil perhitungan pixel model deteksi YOLOv8, delta error diperolah dari selisih antara nilai error aktual dengan error sebelumnya. 


\begin{packed_enum}
	\item Fuzzifikasi: Fuzzifikasi adalah proses mengubah input numerik (crisp) menjadi himpunan fuzzy. Pada tahap ini nilai input error dan delta error diubah menjadi nilai keanggotaan yang terdiri dari "KiriTajam", "Kiri", "Lurus", "Kanan", dan "KananTajam".  seperti pada gambar dibawah ini. Proses fuzzifikasi menggunakan fungsi keanggotaan untuk mentukan derajat keanggotaan nilai error dan delta error dalam setiap himpunan fuzzy yang relevan.
	
	\item Basis Aturan Fuzzy: Aturan fuzzy atau rule fuzzy merupakan kumpulan aturan if-then yang menentukan hubungan antara himpunan fuzzy input dan output.  Setiap rule fuzzy logic berbentuk "jika kondisi, maka aksi" seperti pada gambar dibawah ini. Aturan yang digunakan pada sistem robot semprot memiliki total 25 aturan dengan lima kondisi berbeda yaitu aturan untuk "Kiri Tajam", "Kiri", "Lurus", "Kanan", "Kanan Tajam".
	\item Defuzzifikasi: berfungsi untuk mengubah keluaran fuzzy yang dihasilkan oleh proses agregasi menjadi nilai numerik (crisp) berupa duty cycle. Duty Cycle terdiri dari "Pelan", "Sedang", dan "Cepat".
\end{packed_enum}	

\section{Rancangan Computer Vision}
Rancangan computer\textit{ vision} pada sistem robot semprot ini merupakan pembuatan model deteksi YOLOv8.
\subsection{Rancangan YOLOv8}

\begin{packed_enum}
	\item OpenCV: merupakan library perangkat lunak open source yang dikhususkan untuk pengolahan gamabar dan visual komputer. OpenCV menyediakan berbagai fungsi dan algoritma yang digunakan untuk mendeteksi dan mengenali objek, melakukan analisa gambar dan mengolah vidio. OpenCV juga memiliki integrasi dengan berbagai algoritma kecerdasan buatan, seperti machine learning dan pengenalan pola, yang memungkinkan aplikasi untuk melakukan tugas-tugas yang lebih kompleks.
	\item NumPy: Merupakan library perangkat lunak open source yang menggunakan bahasa pemrograman python, menyediakan dukungan untuk array dan matriks multidimensi bersama dengan koleksi besar fungsi matematika tingkat tinggi untuk bekerja dengan data. 
	\item Detectron2: Merupakan library perangkat lunak open source  yang dikembangkan oleh Facebook AI Research (FAIR) yang bertujuan untuk menyederhanakan dan mempercepat pengembangan sistem deteksi objek dan segmentasi gambar berbasis deep learning. Detectron2 dibangun di atas framework PyTorch dan menyediakan implementasi dari berbagai model dan algoritma terkini dalam bidang penglihatan komputer.
	\item Torch: Merupakan library perangkat lunak open source  yang digunakan untuk pengembangan dan implementasi jaringan saraf serta implementasi bahasa pemrograman Python yang dikenal sebagai PyTorch. Torch digunakan secara luas dalam penelitian kecerdasan buatan dan industri, termasuk dalam pengembangan model untuk pengenalan gambar, pemrosesan bahasa alami, pengenalan suara, dan banyak lagi. Kemampuannya untuk mengintegrasikan dengan baik dengan Python melalui PyTorch telah membuatnya menjadi salah satu pilihan utama bagi para peneliti dan praktisi dalam pengembangan dan penelitian di bidang kecerdasan buatan.
	\item Time: Library "time" adalah salah satu library standar dalam bahasa pemrograman Python yang digunakan untuk melakukan operasi terkait waktu. Library ini menyediakan berbagai fungsi yang memungkinkan pengguna untuk bekerja dengan waktu, termasuk pengukuran waktu, penjadwalan, dan manipulasi waktu.
	\item Roboflow: Roboflow adalah platform pengelolaan data dan pelatihan model untuk proyek visualisasi komputer. Ini dirancang untuk membantu pengembang dan peneliti dalam mengelola, mengolah, dan memanfaatkan data citra untuk melatih model penglihatan komputer dengan lebih efisien. 
\end{packed_enum}

\section{Rancangan Sistem}
Pada penelitian ini terdapat \cref{fig:diagramsistem} rancangan sistem yang berisi alur sistem  kerja robot untuk bernavigasi dengan mendeteksi jalur antra tanaman pada perkebunan cabai.

\begin{figure}[H]
	\centering
	\includegraphics[scale=0.6]{diagram-sistem}
	\caption{Gambar Diagram sistem program}
	\label{fig:diagramsistem}
\end{figure}

Berdsarkan \cref{fig:diagramsistem} diatas, sistem program robot penyemprot pestisida mempunyai 4 proses utama yaitu: 


\subsection{Persiapan dan Pembuatan Dataset}
Persiapan dan pembuatan dilakukan secara mandiri, dataset berupa foto yang diambil di kebun cabai sesuai kebutuhan, \cref{fig:alurdataset} berikut ini tahapan pembuatan dataset menggunakan platform roboflow:

\begin{figure}[H]
	\centering
	\includegraphics[scale=0.6]{alur-dataset}
	\caption{Gambar Tahap Pembuatan Dataset}
	\label{fig:alurdataset}
\end{figure}

\begin{packed_item}
	\item Pengumpulan data: pengumpulan dataset berupa foto perkebunan cabai dengan fokus objek berupa jarak antar tanaman (row crop) sebagai objeknya. Data dalam bentuk foto tersebut akan digunakan untuk melatih algoritma Mask R-CNN.
	\item Anotasi Data: Anotasi atau pelabelan objek berupa jarak antar tanaman (row crop) pada perkebunan cabai, pelabelan pada objek pada foto dilakukan terdiri dari 3 kelas yaitu maju, belok kanan, dan belok kiri dengan id kelas yang berbeda.
	\item pembagian Data: Foto yang sudah dianotasi kemudian dibagi menjadi 3 bagian yaitu, train, validation dan test. Pembagian ini penting untuk menghindari overfitting dan memastikan
	model dapat digeneralisasi dengan baik.
	\item Export dataset: Setelah dataset dibagi menjadi 3 bagian, kemudiaan adalah mengekspor dataset sesuai kebutuhan. Algoritma Mask R-CNN dengan menggunakan Pytorch dan Detectron2 menggunakan dataset dengan format COCO, berisi file gambar dan file json.
\end{packed_item}

\subsection{Training Dataset}
Dataset yang dihasilkan roboflow terdiri dari input dan output yang berpasangan, di mana model Mask R-CNN (machine learning) akan menggunakan input untuk mempelajari pola atau hubungan antara input dan output yang sesuai. Tujuan utama dari menggunakan training dataset adalah untuk menghasilkan model yang dapat membuat prediksi atau keputusan yang akurat ketika diberikan input baru yang tidak pernah dilihat sebelumnya. Training dataset menggunakan platform google colab berbasis cloud yang disediakan oleh Google. 
\subsection{Deploy Trained Model}
Pada tahap ini model yang sudah dilatih menggunakan dataset akan dijalankan pada NVIDIA Jetson Nano. Untuk menjalankan model Mask R-CNN memerlukan persiapan berupa pemasangan beberapa library yang sudah dibahas pada bagain sub bab rancangan software.
\subsection{Pengiriman Data dan Hasil Perhitungan}
Tahap terakhir adalah pengiriman data dari proses image processing pada NVIDIA Jetson Nano menggunakan komunikasi serial. Data dapat berupa perhitungan pixel dari objek yang di segmentasi, dimana hasil perhitungan tersebut akan menentukan kecepatan motor dc untuk bernavigasi.

\section{Rancangan Pengujian}
\subsection{pengujian Algoritma Mask R-CNN pada Gambar}
Pengujian model Mask R-CNN dilakukan dengan menggunakan gambar yang bersumber dari dataset. Suatu Gambar dapat memiliki 1 hingga 2 kelas dari total 3 kelas dengan id yang berbeda. Pengujian dilakukan menggunakan platform Google Colab. Model yang digunakan adalah Mask R-CNN yang dilatih menggunakan frame work Pytorch dan Detectron2. Tujuan pengujian model pada gambar untuk mengetahui performa model untuk mendeteksi objek dan instance segmantation. 

\subsection{Pengujian Algoritma Mask R-CNN di Lapangan}
Tahap terakhir pengujian dilakukan di kebun cabai yang diilustrasikan pada \cref{fig:ilustrasiuji}  Pengujian secara langsung dilakukan dengan tujuan untuk mengetahui fungsinal dan sistem pada robot dapat bekerja dengan baik. Pengujian menggunkan menggunakan WebCam sebagai input NVIDIA Jetson Nano. WebCam berfungsi untuk mendeteksi keadaan lingkungan di sekitar robot. Pada Pengujian ini NCVDIA Jetson Nano akan mengolah data atau melakukan image processing, gambar akan diolah oleh algoritma Mask R-CNN untuk dideteksi objek dan instance segmantation. Hasil image processing inilah akan dihitung lebar pixel, data dari perhitungan tersebut akan dikirim ke ESP32 yang untuk menentukan navigasi robot dengan mengatur putaran motor dc.

\begin{figure}[H]
	\centering
	\includegraphics[scale=0.1]{ilustrasi-pengujian}
	\caption{Gambar Ilustrasi Pengujian Robot}
	\label{fig:ilustrasiuji}
\end{figure}
